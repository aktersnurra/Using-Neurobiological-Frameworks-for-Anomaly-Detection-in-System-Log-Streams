\section*{Acknowledgement}
First and foremost, I would like to thank my industrial supervisor, Armin Catovic, at Ericsson for giving me the opportunity of exploring the fascinating theories of the neocortex, the support during the project, and the interesting discussions we had. 


I'm also grateful to my supervisor Arun Venkitaraman and examiner Joakim Jaldén for reviewing this work.


My journey at KTH started with a preparatory year completely unknowing of what was about to come, but with a determined mind. In the first week or so I was lucky enough to find a group of people to study with. We shared the hardships of the first physics course, hardly being able to draw out the forces acting on a skateboarder in a textbook example correctly. But as we shared these challenging times together our friendships grew, and these people became and will forever be some of my best friends, with whom I shared some of my best times with. So thank you Alan, Carl-Johan, Christian, and Simon.


I would like to especially thank Joakim Lilliesköld for convincing me and my friends into selecting the electrical engineering programme. During my time as a student of electricity I met many lifelong friends in our chapter hall, Tolvan. To them I owe a heartfelt thank you, as they made the life as an engineering student more enjoyable.


To my siblings; Daniel, Sara, Johan, and Hanna, you all are true sources of inspiration, without you I would not be where I am today. 


Lastly, to my wonderful parents, I can not express in words how much your support and belief in me has meant. Thank you for always being there for me; for the unconditional love and support.\\\\


\begin{table}[hb]
\begin{tabular}{lp{6.67cm}llll}
& & & & \textit{Gustaf Rydholm,} \\
& & & & Stockholm, 2018
\end{tabular}
\end{table}




