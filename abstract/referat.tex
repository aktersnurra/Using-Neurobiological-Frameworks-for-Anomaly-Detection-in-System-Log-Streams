\begin{foreignabstract}{swedish}
    \textit{Artificiell Intelligens} (AI) har visat enorm potential och är förutspådd att revolutionera hela industrier och introducera en ny industriell era. Men, mestadelen av dagens AI är antingen optimerings algoritmer, eller som med deep learning, en grovt förenklad abstraktion av däggdjurshjärnans funktionalitet. År 1978 föreslog dock Vernon Mountcastle en ny järv idé om hjärnbarken funktionalitet. Dessa idéer har i sin tur varit en inspiration för teorier om sann maskinintelligens. 
    
    I detta examensarbete studerar vi en sådan teori, kallad \textit{Hierarchical Temporal Memory} (HTM). Detta ramverk använder vi sedan för att bygga en maskininlärningsmodel, som kan hitta och klassificera fel och icke-fel i systemloggar från komponent baserad mjukvara utvecklad av Ericsson. Vi jämför sedan resultaten med en existerade maskininlärningsmodel, kallad \textit{Linnaeus}, som använder sig av klassiska maskininlärningsmetoder. HTM-modellen visar lovande resultat, där HTM-modellen klassificera systemloggar korrekt med snarlika resultat som Linnaeus. HTM-modellen anses vara en lovade algoritm för framtida evalueringar på ny data då den för det första kan lära sig via \textit{``one-shot learning''}, och för det andra inte är beräkningstung modell. 
\end{foreignabstract}
\endinput