%%
%% This is file `kth-abs.tex',
%% generated with the docstrip utility.
%%
%% The original source files were:
%%
%% kthesis.dtx  (with options: `abstract')
%% 
%% IMPORTANT NOTICE:
%% 
%% For the copyright see the source file.
%% 
%% Any modified versions of this file must be renamed
%% with new filenames distinct from kth-abs.tex.
%% 
%% For distribution of the original source see the terms
%% for copying and modification in the file kthesis.dtx.
%% 
%% This generated file may be distributed as long as the
%% original source files, as listed above, are part of the
%% same distribution. (The sources need not necessarily be
%% in the same archive or directory.)
\begin{abstract}
  \textit{Artificial Intelligence} (AI) has shown enormous potential, and is predicted to be a prosperous field that will likely revolutionise entire industries and bring forth a new industrial era. However, most of today's AI is either, as in deep learning, an oversimplified abstraction of how an actual mammalian brains neural network function, or methods sprung from mathematics. But, with the foundation of the bold ideas of Vernon Mountcastle stated in 1978 about the neocortical functionality, new frameworks for creating true machine intelligence have been developed, and continues to be. 
  
  In this thesis, we study one such theory, called \textit{Hierarchical Temporal Memory} (HTM). We use this framework to build a machine learning model in order to solve the task of detecting and classifying anomalies in system logs belonging to Ericsson's component based architecture applications. The results are then compared to an existing classifier, called \textit{Linnaeus}, which uses classical machine learning methods. The HTM model is able to show promising capabilities of classifying system log sequences with similar results compared with the Linnaeus model. The HTM model is an appealing alternative, due to the limited need of computational resources and the algorithms ability to effectively learn with \textit{``one-shot learning''}.

\end{abstract}
\endinput
%%
%% End of file `kth-abs.tex'.
